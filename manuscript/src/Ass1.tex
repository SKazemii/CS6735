\documentclass[12pt]{article}

\usepackage{Preamble}

\input{acronyms.tex}

\title{Programming Project}				% Title
\author{Saeed Kazemi}				% Author
\date{\today}						% Date

\makeatletter
\let\theauthor\@author
\let\thedate\@date
\let\thetitle\@title
\makeatother
\begin{document}

%%%%%%%%%%%%%%%%%%%%%%%%%%%%%%%%%%%%%%%%%%%%%%%%%%%%%%%%%%%%%%%%%


\begin{titlepage}
	\centering
    \vspace*{0.4 cm}
    \includegraphics[scale = 0.5]{figures/unb.jpg}\\[1.0 cm]	% University Logo
    \textsc{\LARGE \newline\newline University of New Brunswick}\\[1.8 cm]	% University Name
	\textsc{\Large Machine Learning and Data Mining\\(CS 6735)}\\[0.5 cm]				% Course Code
	\rule{\linewidth}{0.2 mm} \\[0.4 cm]
	{ \huge \bfseries \thetitle}\\
	\rule{\linewidth}{0.2 mm} \\[1.5 cm]
	
	\begin{minipage}{0.5\textwidth}
		\begin{flushleft} \large
			\emph{Professor:}\\
			Huajie Zhang\\
            Computer Science\\
			\end{flushleft}
			\end{minipage}~
			\begin{minipage}{0.5\textwidth}
            
			\begin{flushright} \large
			\emph{Author:} \\
			Saeed Kazemi\\ (3713280)\\

		\end{flushright}
        
	\end{minipage}\\[1 cm]
	
	
    \thedate
    
    
    
	
\end{titlepage}

%%%%%%%%%%%%%%%%%%%%%%%%%%%%%%%%%%%%%%%%%%%%%%%%%%%%%%%%%%%%%%%%%



\tableofcontents
\pagebreak
\newpage
\vspace{2cm}

\section*{Introduction}
\vspace{2cm}
This project has focused on comparing and discussing five various machine learning algorithms based on the experimental results on the five different datasets. Furthermore, this research will be implemented in Python, and the source codes are available on the GitHub \href{https://github.com/SKazemii/CS6735}{ repository}. 

The rest of this assignment is organized as follows. Section 1 provides some information about the dataset along with techniques that are used for pre-processing. Then, in Section 2, the machine learning algorithms will be reviewed. Finally, the results are discussed in Section 3.
%%%%%%%%%%%%%%%%%%%%%%%%%%%%%%%%%%%%%%%%%%%%%%%%%%%%%%%%%%%%%%%%%
%================================================================
\newpage
\section*{The Questions}
{\small
Conduct an experimental study on the following machine learning algorithms: (1) ID3; (2) Adaboost on ID3; (3) Random Forest; (4) Naïve Bayes; (5) K-nearest neighbors (kNN).


\begin{enumerate}
\item Implement the five algorithms using Java or Python.
\item Evaluate your implementation on the datasets in data.zip (downloadable from course website) using 10 times 5-fold cross-validation, and report the average accuracy and standard deviation. All datasets are for UCI machine learning repository. You can check the detailed descriptions from this 
\href{http://www.ics.uci.edu/~mlearn/MLRepository.html}{ link.}

\begin{itemize}
    \item For breast cancer data see this \href{http://archive.ics.uci.edu/ml/datasets/Breast+Cancer+Wisconsin+\%28Diagnostic\%29}{ link.}
    \item For car data see this \href{http://archive.ics.uci.edu/ml/datasets/Car+Evaluation}{ link.} 
    \item For ecoli data see this \href{http://archive.ics.uci.edu/ml/datasets/Ecoli}{ link.}
    \item For letter recognition data see this \href{http://archive.ics.uci.edu/ml/datasets/Letter+Recognition}{ link.}
    \item For mushroom data see this \href{http://archive.ics.uci.edu/ml/datasets/Mushroom}{ link.} 
\end{itemize}

For each data set, there is a target variable, the one your model predicts. The following are the target variable for each data set.

\begin{itemize}
    \item Mushroom: first column (e, p)
    \item Letter: first column (A, B, ...)
    \item Ecoli: last column (cp, im, ..)
    \item Car: last column (acc, uacc, ..)
    \item Breast-cancer: last column (2, 4)
\end{itemize}

\item Compare and discuss your algorithms (implementations) based on your experimental results. 

\textbf{Submission:}
\begin{enumerate}
\item Hand in a report of your experimental study via Desire2Learning, including:
\begin{enumerate}
\item Description of the learning algorithms you implement.
\item Description of the datasets you use (number of examples, number of attribute, number of classes, type of attributes, etc.).
\item Technical details of your implementation: pre-processing of data sets (discretization, etc.), parameter setting, etc. 
\item Design of your programming implementation (data structures, overall program structure).
\item Report and analysis of your experimental results. 
\end{enumerate}

\item Submit your code via Desire2Learning.
\end{enumerate}
\end{enumerate}

}

  


%%%%%%%%%%%%%%%%%%%%%%%%%%%%%%%%%%%%%%%%%%%%%%%%%%%%%%%%%%%%%%%%%
%%%%%%%%%%%%%%%%%%%%%%%% Question 1 %%%%%%%%%%%%%%%%%%%%%%%%%%%%%
%%%%%%%%%%%%%%%%%%%%%%%%%%%%%%%%%%%%%%%%%%%%%%%%%%%%%%%%%%%%%%%%%
\newpage
\vspace{2cm}
\section{ Datasets and pre-processing}
\vspace{2cm}
In this section, we will review the datasets as well as the pre-processing techniques. Dataset can be downloaded from the Desire2Learning website. 
Before describing the dataset, let look at the discretization algorithm.



%Description of the datasets you use (number of examples, number of attribute, number of classes, type of attributes, etc.).
%Technical details of your implementation: pre-processing of data sets (discretization, etc.), parameter setting, etc.
\subsection{Discretization algorithm}
\label{ses:Discretization}
Since some of algorithms accept only discreated values, I have used the equal width binning algorithm to discretize the continuous data to categorical version. This method to partition the range of features into k equal-width distances. The interval width (w) is calculated by the below equation.

$$
w = \frac{[ max(feature)-min(feature)] }{k}
$$

Then, the $i^{th}$ interval range can be found by using this equation.

$$
[min(feature) + (i-1)w,~~ min(feature) + iw] \mbox{~~where~~}  i = 1, 2, 3, \dots, k
$$

Now, we can scan each features values and if it is in the $i^{th}$ interval range, we replace it with $i$ value. As a result, the new feature set has only $i$ different value.

Breast cancer, Ecoli and letter dataset have the continuous attributes, and I have discretized them for using ID3, Random Forests, and Adaboost classifier. 

\begin{table}[H]
\centering
\caption{The equal width binning function for discretizing the continious data to categorical version.}
\begin{lstlisting}
def discr(x, k=5):
    w = (np.max(x) - np.min(x)) / k
    bins = [np.min(x) + (i) * w for i in range(k)]
    return np.digitize(x, bins=bins, right=False)
\end{lstlisting}
\end{table}





\subsection{Categorical to numerical values}
Car and Mushroom dataset are two data set that have categorical values. To converting these two datasets to a contentious data, I have replaced the categorical values with a numerical values. 




\subsection{Breast Cancer Wisconsin (Diagnostic) Dataset}

This dataset was included real data about breast cancer. In the CSV file, there are 11 columns. The first column, which is a seven-digit number, indicates the patient ID. Also, the last column is our target. Other columns were real-valued features that are computed from each cell nucleus. 

For reading this dataset, we used a Pandas DataFrame. Since some values in the dataset were a question mark, we replaced those with a NaN value. After that, we drop all rows that had NaN values. The final size of the dataset is 682 rows $\times$ 10 columns. Based on the Question, this dataset has two class values, 2 and 4.








%Description of the datasets you use (number of examples, number of attribute, number of classes, type of attributes, etc.).
%Technical details of your implementation: pre-processing of data sets (discretization, etc.), parameter setting, etc.

\subsection{Car Evaluation Dataset}

Car Evaluation Database was derived from a simple hierarchical decision model. This dataset has four class values named unacc, acc, good, vgood, and it is located in the final column. Other features are buying, maint, doors, persons, safety and lug\_boot. These features have categorical values. The table below shows the attribute information.  


\begin{table}[H]
\centering
\caption{The attribute information of Car Evaluation dataset.}
\label{tab:tab_1}
\begin{tabular}{rl}
\toprule
Column Names & their values \\
\midrule
buying               &   vhigh, high, med, low \\
maint & vhigh, high, med, low\\
doors & 2, 3, 4, 5more\\
persons & 2, 4, more\\
lug\_boot & small, med, big\\
safety & low, med, high\\
Classes & unacc, acc, good, vgood\\
\bottomrule
\end{tabular}

\end{table}

In addition, there is no missing value in the dataset. Therefore, the final size of this dataset is 6 $\times$ 1728. Furthermore, in order to use some algorithms like kNN, we changed the categorical values to numerical values by means of the replace method in Pandas DataFrame. 




\subsection{Ecoli Dataset}
This data contains protein localization sites. The size of the dataset is 336 $\times$ 8, and there is no missing value. Table \ref{tab:tab_2} indicates the attribute information of this dataset. Furthermore, All features have a real value.

\begin{table}[H]
\centering
\caption{The attribute information of Ecoli dataset.}
\label{tab:tab_2}
\begin{tabular}{rll}
\toprule
\# & Attribute Names & description \\
\midrule
1 & Sequence Name & Accession number for the SWISS-PROT database\\
2 & mcg & McGeoch's method for signal sequence recognition.\\
3 & gvh & von Heijne's method for signal sequence recognition.\\
4 & lip & von Heijne's Signal Peptidase II consensus sequence score. \\
5 & chg & Presence of charge on N-terminus of predicted lipoproteins.\\
6 & aac & \vtop{\hbox{\strut score of discriminant analysis of the amino acid}\hbox{\strut  content of outer membrane and periplasmic proteins.}}\\
%6 & aac & \makecell{ \\the amino acid content of outer membrane and periplasmic proteins.}\\
7 & alm1 & \vtop{\hbox{\strut score of the ALOM membrane spanning}\hbox{\strut  region prediction program.}}\\
8 & alm2 & \vtop{\hbox{\strut score of ALOM program after excluding putative}\hbox{\strut cleavable signal regions from the sequence.}}\\
\bottomrule
\end{tabular}
\end{table}

The last column shows the classes of this dataset which takes eight different values.




\subsection{Letter Recognition Dataset}

The objective of this dataset is to classify each of the 26 capital letters in the English alphabet. The character images were based on 20 different fonts, and each letter within these 20 fonts was randomly distorted to produce a file of 20,000 unique samples. Each sample was converted into 16 numerical attributes. These values are an integer number between 0 and 15. 

\begin{table}[H]
\centering
\caption{The attribute information of letter recognition dataset.}
\label{tab:tab_3}
\input{tables/tab_3.tex}
\end{table}

The size of the dataset is 20000 $\times$ 16. The first column also shows the class of sample that is a capital letter. This dataset has no missing values. In addition, all classes are replaced with a numerical value as a preprocessing step.

\subsection{Mushroom Dataset}

This data set includes descriptions of hypothetical samples corresponding to 23 species of gilled mushrooms in the Agaricus and Lepiota Family. Each species is identified as definitely edible (e) or poisonous (p). There are about 22 categorical features for each sample in the dataset. Some samples have a character "?" which indicates the dataset has some missing value. Also, all categorical values are replaced with numerical values. The size of the dataset is 8124 $\times$ 16.



%%%%%%%%%%%%%%%%%%%%%%%%%%%%%%%%%%%%%%%%%%%%%%%%%%%%%%%%%%%%%%%%%
%%%%%%%%%%%%%%%%%%%%%%%% Question 2 %%%%%%%%%%%%%%%%%%%%%%%%%%%%%
%%%%%%%%%%%%%%%%%%%%%%%%%%%%%%%%%%%%%%%%%%%%%%%%%%%%%%%%%%%%%%%%%
\newpage
\vspace{2cm}
\section{ Implemented algorithms}
\vspace{2cm}

In this section, we will review the algorithms as well as implemented techniques. Furthermore, to implementing k-fold cross-validation, we implemented a classifier class which the Parent’s class of other algorithms is.
In the classifier class, we have three methods, kfold\_split(), accuracy(), and fit(). Table \ref{tab:tab_classifier} indicates the methods and attributes of this class. 


\begin{table}[H]
\centering
\caption{The methods and attributes of classifier class.}
\label{tab:tab_classifier}
{\small
\begin{tabular}{llll}
\toprule
\multicolumn{4}{|c|}{Class Name: classifiers} \\
\midrule
\multicolumn{2}{|c|}{parent: -} &\multicolumn{2}{|c|}{Child: KNN, NB, KNN, RF} \\\bottomrule


\multicolumn{4}{c}{} \\\bottomrule
\multicolumn{4}{|c|}{Attributes} \\\midrule
\multicolumn{1}{|r}{n\_folds: } & \multicolumn{3}{l|}{the number of folds } \\\bottomrule



\multicolumn{4}{c}{} \\\bottomrule
\multicolumn{4}{|c|}{Methods} \\\midrule
\multicolumn{2}{|r}{kfold\_split(dataset, n\_folds) } & \multicolumn{2}{l|}{split the dataset to n\_folds} \\
\multicolumn{2}{|r}{accuracy(y\_true, y\_pred) } & \multicolumn{2}{l|}{compute the accuracy} \\
\multicolumn{2}{|r}{accuracy = fit() } & \multicolumn{2}{l|}{train the algorithm} \\\midrule


\multicolumn{1}{|c|}{Parameters}
& \multicolumn{1}{|r}{dataset:}& \multicolumn{2}{l|}{array-like of shape (n\_samples, n\_features+target)}\\
\multicolumn{1}{|l|}{} & \multicolumn{1}{|r}{n\_folds}& \multicolumn{2}{l|}{int, default=5}\\
\multicolumn{1}{|l|}{} & \multicolumn{1}{|r}{y\_true}& \multicolumn{2}{l|}{list-like of shape (n\_samples,)}\\
\multicolumn{1}{|l|}{} & \multicolumn{1}{|r}{y\_pred}& \multicolumn{2}{l|}{list-like of shape (n\_samples,)}\\\midrule

\multicolumn{1}{|c|}{Returns} & \multicolumn{1}{|r}{accuracy}& \multicolumn{2}{l|}{accuracy score of each k\_fold in percentage}\\




\bottomrule
\end{tabular}}
\end{table}

I have implemented the k\_fold algorithm as follows:
\begin{enumerate}
    \item Find the length of each fold by dividing the dataset length into k\_fold.
    \item Select an index randomly, and then copy the selected index to the new set by means of the pop method in Pandas.
    \item Check the length of the new set. If it is less than part (1), repeat part (2), otherwise make another set.
\end{enumerate}











\subsection{K-nearest neighbors algorithm (KNN)}

For using this algorithm, we need to make an instance of the KNN class. This instance calls the \_\_init\_\_ method to set some attributes for the algorithm. After that, with the fit method, we train the algorithm. The table below shows a piece of code used for making an instance of the KNN class. Moreover, table \ref{tab:tab_knn} illustrates the methods and attributes of this class. Furthermore, the k parameter must be an odd number. 


\begin{table}[H]
\centering
\caption{Making an instance of KNN class.}
\begin{lstlisting}
for _ in range(10):
    knn = KNN(n_folds=5, dataset=dataset, k_neighbor=3)
    accuracy = knn.fit()
    acc.append(sum(accuracy) / len(accuracy))
\end{lstlisting}
\end{table}




\begin{table}[H]
\centering
\caption{The methods and attributes of KNN class.}
\label{tab:tab_knn}
\input{manuscript/src/tables/tab_knn}
\end{table}

I have implemented the kNN algorithm as follows:

\begin{enumerate}
    \item For each sample in the test data, calculate the distances between the current sample and all samples in the training set (with knn() method)
    \item Sort the distances list in ascending order and find the class of the top k\_neighbor (with find\_response() method).
    \item Pick and return the first and most repeated label from the previous collection as a label of test example (with find\_response() method).
    
\end{enumerate}













\subsection{Naive Bayes algorithm}

For implementing this algorithm, we wrote the NB class. Also, we consider normal distribution for implementation. Table \ref{tab:tab_nb} illustrates the methods and attributes of this class.

\begin{table}[H]
\centering
\caption{The methods and attributes of NB class.}
\label{tab:tab_nb}
{\small
\begin{tabular}{llll}
\toprule
\multicolumn{4}{|c|}{Class Name: NB} \\
\midrule
\multicolumn{2}{|c|}{parent: classifiers} &\multicolumn{2}{|c|}{Child: -} \\\bottomrule


\multicolumn{4}{c}{} \\\bottomrule
\multicolumn{4}{|c|}{Attributes} \\\midrule
\multicolumn{1}{|r}{n\_folds: } & \multicolumn{3}{l|}{the number of folds } \\
\multicolumn{1}{|r}{dataset: } & \multicolumn{3}{l|}{the train dataset (the last column is the target) } \\\bottomrule


\multicolumn{4}{c}{} \\\bottomrule
\multicolumn{4}{|c|}{Methods} \\\midrule
\multicolumn{2}{|r}{\_\_init\_\_(dataset, n\_folds, k\_neighbor) } & \multicolumn{2}{l|}{Constructor to initialise the attributes of the class} \\
\multicolumn{2}{|r}{naivebayes(trainset, testset) } & \multicolumn{2}{l|}{Main loop of class} \\
\multicolumn{2}{|r}{predict(models, test\_row) } & \multicolumn{2}{l|}{Return the prediction value} \\
\multicolumn{2}{|r}{model\_classes(dataset) } & \multicolumn{2}{l|}{Set a Gaussian model for each class} \\
\multicolumn{2}{|r}{find\_pdf(x, mean, stdev) } & \multicolumn{2}{l|}{Calculate probability} \\ \midrule


\multicolumn{1}{|c|}{Parameters}
& \multicolumn{1}{|r}{dataset:}& \multicolumn{2}{l|}{array-like of shape (n\_samples, n\_features+target)}\\
\multicolumn{1}{|l|}{} & \multicolumn{1}{|r}{n\_folds}& \multicolumn{2}{l|}{int, default=5}\\
\multicolumn{1}{|l|}{} & \multicolumn{1}{|r}{k\_neighbor}& \multicolumn{2}{l|}{an odd integer, default=3}\\
\multicolumn{1}{|l|}{}& \multicolumn{1}{|r}{tr\_set:}& \multicolumn{2}{l|}{array-like of shape (n\_samples, n\_features+target)}\\
\multicolumn{1}{|l|}{}& \multicolumn{1}{|r}{te\_set:}& \multicolumn{2}{l|}{array-like of shape (n\_samples, n\_features+target)}\\\midrule

\multicolumn{1}{|c|}{Returns} & \multicolumn{1}{|r}{pred:}& \multicolumn{2}{l|}{predicted classes}\\




\bottomrule
\end{tabular}}
\end{table}

The table below shows a piece of code used for making an instance of the NB class.

\begin{table}[H]
\centering
\caption{Making an instance of NB class.}
\begin{lstlisting}
for _ in range(10):
    nb = NB(n_folds=5, dataset=dataset)
    accuracy = nb.fit()
    acc.append(sum(accuracy) / len(accuracy))
\end{lstlisting}
\end{table}


I have implemented the Naive Bayes algorithm as follows:
\begin{enumerate}
    \item Split training set by class value (model\_classes())
    \item Find the mean and std of each attribute in each split dataset by their class (model\_classes()).
    \item Calculate the class probability for each test sample (find\_pdf()). Combine probability of each feature (predict()).
    \item Compare probability for each class. Return the class label which has max probability (predict()).
    
\end{enumerate}















\subsection{Iterative Dichotomiser 3 algorithm (ID3)}
We wrote the ID3 class to implement this algorithm.  Table \ref{tab:tab_id3} illustrates the methods and attributes of this class. This algorithm calls yourself like a recursive function. Moreover, the dataset must be a categorical data. Therefore, I have used 5-bin discretization for all attributes in Ecoli, Breast Cancer, and Letter datasets.

\begin{table}[H]
\centering
\caption{The methods and attributes of ID3 class.}
\label{tab:tab_id3}
{\small
\begin{tabular}{llll}
\toprule
\multicolumn{4}{|c|}{Class Name: ID3} \\
\midrule
\multicolumn{2}{|c|}{parent: classifiers} &\multicolumn{2}{|c|}{Child: RF} \\\bottomrule


\multicolumn{4}{c}{} \\\bottomrule
\multicolumn{4}{|c|}{Attributes} \\\midrule
\multicolumn{1}{|r}{n\_folds: } & \multicolumn{3}{l|}{the number of folds } \\
\multicolumn{1}{|r}{dataset: } & \multicolumn{3}{l|}{the train dataset (the last column is the target) } \\\bottomrule


\multicolumn{4}{c}{} \\\bottomrule
\multicolumn{4}{|c|}{Methods} \\\midrule
\multicolumn{2}{|r}{\_\_init\_\_(dataset, n\_folds, attr\_names) } & \multicolumn{2}{l|}{constructor to initialise the attributes of the class} \\
\multicolumn{2}{|r}{id3(trainset, testset) } & \multicolumn{2}{l|}{Build the ID3 algorithm} \\
\multicolumn{2}{|r}{decision\_tree(trainset, attr\_names) } & \multicolumn{2}{l|}{Build the decision tree} \\
\multicolumn{2}{|r}{split(trainset, arc, val) } & \multicolumn{2}{l|}{Split the dataset based on an attribute} \\
\multicolumn{2}{|r}{attr\_selection(trainset, attr\_names) } & \multicolumn{2}{l|}{Find the best attribute for the split } \\
\multicolumn{2}{|r}{entropy(data) } & \multicolumn{2}{l|}{Compute the entropy} \\
\multicolumn{2}{|r}{test(test\_set, tree, attr\_names)} & \multicolumn{2}{l|}{Calculate the prediction accuracy} \\
\multicolumn{2}{|r}{predict(models, test\_row) } & \multicolumn{2}{l|}{Predict an unseen instance} \\ \midrule


\multicolumn{1}{|c|}{Parameters}
& \multicolumn{1}{|r}{dataset:}& \multicolumn{2}{l|}{array-like of shape (n\_samples, n\_features+target)}\\
\multicolumn{1}{|l|}{} & \multicolumn{1}{|r}{n\_folds:}& \multicolumn{2}{l|}{int, default=5}\\
\multicolumn{1}{|l|}{} & \multicolumn{1}{|r}{names:}& \multicolumn{2}{l|}{a list of feature names}\\\midrule

\multicolumn{1}{|c|}{Returns} & \multicolumn{1}{|r}{pred:}& \multicolumn{2}{l|}{predicted classes}\\




\bottomrule
\end{tabular}}
\end{table}
To use this algorithm, you need to make an instance first. The table below shows a piece of code used for doing it.

\begin{table}[H]
\centering
\caption{Making an instance of ID3 class.}
\begin{lstlisting}
for _ in range(10):
    id3 = ID3(n_folds=5, dataset=dataset, names=names)
    accuracy = id3.fit()
    acc.append(sum(accuracy) / len(accuracy))
\end{lstlisting}
\end{table}


I have implemented the ID3 algorithm as follows:
\begin{enumerate}
    \item Create a node. If all samples belong to the same class, the node is marked as a leaf node and returns the class.

    \item For each feature in the data set, calculate the Information Gain. Then split feature that has the maximum information gain.
        
    \item For each value in the split attribute, extend the corresponding branch, and divide samples according to the feature value.
    
    \item Use the same procedure, recursion from the top down until one of the following three conditions is met and the recursion stops. If there are still samples belonging to different categories in the leaf node, the category containing the most samples is selected as the classification of the leaf node.


    \begin{itemize}
        \item All samples belong to the same class.
        \item All samples in the training set were classified.
        \item All features were executed once as split features.
    \end{itemize}
\end{enumerate}












\subsection{Random Forests algorithm}
We used the ID3 class to implement the random forests algorithm. Table \ref{tab:tab_rf} illustrates the methods and attributes of this class.  To converting continuous values to discreated version, I have used 5-bin discretization for all attributes in Ecoli, Breast Cancer, and Letter datasets.

\begin{table}[H]
\centering
\caption{The methods and attributes of Random Forests class.}
\label{tab:tab_rf}
{\small
\begin{tabular}{llll}
\toprule
\multicolumn{4}{|c|}{Class Name: RF} \\
\midrule
\multicolumn{2}{|c|}{parent: ID3} &\multicolumn{2}{|c|}{Child: -} \\\bottomrule


\multicolumn{4}{c}{} \\\bottomrule
\multicolumn{4}{|c|}{Attributes} \\\midrule
\multicolumn{1}{|r}{n\_folds: } & \multicolumn{3}{l|}{the number of folds } \\
\multicolumn{1}{|r}{num\_trees: } & \multicolumn{3}{l|}{the number of trees } \\
\multicolumn{1}{|r}{dataset: } & \multicolumn{3}{l|}{the train dataset (the last column is the target) } \\\bottomrule


\multicolumn{4}{c}{} \\\bottomrule
\multicolumn{4}{|c|}{Methods} \\\midrule
\multicolumn{2}{|r}{\_\_init\_\_(data, n\_folds, attrNames, numTrees) } & \multicolumn{2}{l|}{constructor to initialise the attributes of the class} \\
\multicolumn{2}{|r}{pred = RF(trainset, testset) } & \multicolumn{2}{l|}{Build the Random forest algorithm} \\
\multicolumn{2}{|r}{voting(prediction) } & \multicolumn{2}{l|}{Find the most common result from trees} \\
\multicolumn{2}{|r}{bootstrapping(data) } & \multicolumn{2}{l|}{Make a new dataset randomly from data} \\ \midrule


\multicolumn{1}{|c|}{Parameters}
& \multicolumn{1}{|r}{dataset:}& \multicolumn{2}{l|}{array-like of shape (n\_samples, n\_features+target)}\\
\multicolumn{1}{|l|}{} & \multicolumn{1}{|r}{n\_folds:}& \multicolumn{2}{l|}{int, default=5}\\
\multicolumn{1}{|l|}{} & \multicolumn{1}{|r}{numTrees:}& \multicolumn{2}{l|}{int, default=5}\\
\multicolumn{1}{|l|}{} & \multicolumn{1}{|r}{attrNames:}& \multicolumn{2}{l|}{a list of feature names}\\\midrule

\multicolumn{1}{|c|}{Returns} & \multicolumn{1}{|r}{pred:}& \multicolumn{2}{l|}{predicted classes}\\




\bottomrule
\end{tabular}}
\end{table}

I have implemented the Random Forests algorithm as follows:
\begin{enumerate}
    \item For Each tree do:
    \begin{enumerate}
        \item Create a bagging dataset.
        \item Train a ID3 algorithm based on the new dataset.
        \item Test the tree with the test set.
    \end{enumerate}
    \item Vote between the results of algorithms to find the final result.
    
\end{enumerate}

The table below shows a piece of code that is used for making an instance. 

\begin{table}[H]
\centering
\caption{Making an instance of Random Forests class.}
\begin{lstlisting}
for _ in range(10):
    rf = RF(n_folds=5, dataset=dataset, names=names)
    accuracy = rf.fit()
    acc.append(sum(accuracy) / len(accuracy))
\end{lstlisting}
\end{table}






\subsection{Adaboost algorithm}
We used the binary classification (stump) to implement the Adaboost algorithm. Table \ref{tab:tab_ab} illustrates the methods and attributes of this class. Moreover, the implemented algorithm could do binary classification. Also, the dataset must be a categorical data. So, I have used 5-bin discretization for all attributes in Ecoli, Breast Cancer, and Letter datasets.

\begin{table}[H]
\centering
\caption{The methods and attributes of Adaboost class.}
\label{tab:tab_ab}
{\small
\begin{tabular}{llll}
\toprule
\multicolumn{4}{|c|}{Class Name: AB} \\
\midrule
\multicolumn{2}{|c|}{parent: classifiers} &\multicolumn{2}{|c|}{Child: -} \\\bottomrule


\multicolumn{4}{c}{} \\\bottomrule
\multicolumn{4}{|c|}{Attributes} \\\midrule
\multicolumn{1}{|r}{n\_folds: } & \multicolumn{3}{l|}{the number of folds } \\
\multicolumn{1}{|r}{n\_clfs: } & \multicolumn{3}{l|}{the number of weak classifiers } \\
\multicolumn{1}{|r}{dataset: } & \multicolumn{3}{l|}{the train dataset (the last column is the target) } \\\bottomrule


\multicolumn{4}{c}{} \\\bottomrule
\multicolumn{4}{|c|}{Methods} \\\midrule
\multicolumn{2}{|r}{\_\_init\_\_(data, n\_folds, n\_clfs) } & \multicolumn{2}{l|}{constructor to initialise the attributes of the class} \\
\multicolumn{2}{|r}{pred = AB(trainset, testset) } & \multicolumn{2}{l|}{Build the Adaboost algorithm} \\
\multicolumn{2}{|r}{ABoost(X, y) } & \multicolumn{2}{l|}{Find the weak lassifiers } \\
\multicolumn{2}{|r}{predict(data) } & \multicolumn{2}{l|}{Predict the classes} \\ \midrule


\multicolumn{1}{|c|}{Parameters}
& \multicolumn{1}{|r}{dataset:}& \multicolumn{2}{l|}{array-like of shape (n\_samples, n\_features+target)}\\
\multicolumn{1}{|l|}{} & \multicolumn{1}{|r}{n\_folds:}& \multicolumn{2}{l|}{int, default=5}\\
\multicolumn{1}{|l|}{} & \multicolumn{1}{|r}{n\_clfs:}& \multicolumn{2}{l|}{int, default=5}\\\midrule

\multicolumn{1}{|c|}{Returns} & \multicolumn{1}{|r}{pred:}& \multicolumn{2}{l|}{predicted classes}\\




\bottomrule
\end{tabular}}
\end{table}
We used this Adaboost algorithm flow for this project:
\begin{enumerate}
    \item Initially set uniform example weights.
    \item For Each weak classifier do:
    \begin{enumerate}
        \item Greedy search to find best threshold and feature.
        \item Find the lowest error as sum of weights of misclassified samples 
        \item Store the best configuration that has lowest error as a first weak classifier
        \item Calculate predictions and update weights
        \item Normalize to weights
    \end{enumerate}

\end{enumerate}

The table below shows a piece of code that is used for making an instance. 

\begin{table}[H]
\centering
\caption{Making an instance of Adaboost class.}
\begin{lstlisting}
for _ in range(10):
    adaboost = AB(n_folds=5, dataset=dataset, n_clf=5)
    accuracy = adaboost.fit()
    acc.append(sum(accuracy) / len(accuracy))
\end{lstlisting}
\end{table}
%%%%%%%%%%%%%%%%%%%%%%%%%%%%%%%%%%%%%%%%%%%%%%%%%%%%%%%%%%%%%%%%%
%%%%%%%%%%%%%%%%%%%%%%%% Question 3 %%%%%%%%%%%%%%%%%%%%%%%%%%%%%
%%%%%%%%%%%%%%%%%%%%%%%%%%%%%%%%%%%%%%%%%%%%%%%%%%%%%%%%%%%%%%%%%
\newpage
\vspace{2cm}
\section{ Results and Discussion}
\vspace{2cm}

In this section, we reviewed the result of implemented algorithms on datasets.



%Report and analysis of your experimental results. Compare and discuss your algorithms (implementations) based on your experimental results..



\subsection{K-nearest neighbors algorithm (KNN)}
This algorithm is very simple and easy to implement. It has only one hyper-parameter (k\_neighbor), and we considered it as 3 for all datasets. The algorithm got significantly slower for the letter and mushroom dataset because the volume of data increased. Furthermore, this algorithm takes the numerical values and since Car and Mushroom dataset have the categorical attributes, I have converted them to numerical version by the replace method in Pandas library. 

\begin{table}[H]
\centering
\caption{The accuracy of KNN on cancer dataset.}
\label{tab:tab_knn_canc}
\begin{tabular}{lr}
\toprule
{} &  Accuracy KNN on cancer dataset \\
\midrule
The 1st run         &                           97.65 \\
The 2nd run         &                           97.21 \\
The 3rd run         &                           97.21 \\
The 4th run         &                           97.06 \\
The 5th run         &                           97.65 \\
The 6th run         &                           97.35 \\
The 7th run         &                           97.50 \\
The 8th run         &                           97.06 \\
The 9th run         &                           96.62 \\
The 10th run        &                           97.06 \\
Average accuracy:   &                           97.24 \\
Standard deviation: &                            0.30 \\
\bottomrule
\end{tabular}

\end{table}

\begin{table}[H]
\centering
\caption{The accuracy of KNN on cars dataset.}
\label{tab:tab_knn_cars}
\begin{tabular}{lr}
\toprule
{} &  Accuracy KNN on cars dataset \\
\midrule
The 1st run         &                         91.77 \\
The 2nd run         &                         92.29 \\
The 3rd run         &                         92.06 \\
The 4th run         &                         92.00 \\
The 5th run         &                         92.17 \\
The 6th run         &                         91.19 \\
The 7th run         &                         91.01 \\
The 8th run         &                         92.06 \\
The 9th run         &                         92.46 \\
The 10th run        &                         91.42 \\
Average accuracy:   &                         91.84 \\
Standard deviation: &                          0.46 \\
\bottomrule
\end{tabular}

\end{table}

\begin{table}[H]
\centering
\caption{The accuracy of KNN on ecol dataset.}
\label{tab:tab_knn_ecol}
\begin{tabular}{lr}
\toprule
{} &  Accuracy KNN on ecol dataset \\
\midrule
The 1st run         &                         82.99 \\
The 2nd run         &                         83.58 \\
The 3rd run         &                         85.37 \\
The 4th run         &                         83.88 \\
The 5th run         &                         84.78 \\
The 6th run         &                         85.07 \\
The 7th run         &                         84.18 \\
The 8th run         &                         86.87 \\
The 9th run         &                         84.18 \\
The 10th run        &                         84.18 \\
Average accuracy:   &                         84.51 \\
Standard deviation: &                          1.03 \\
\bottomrule
\end{tabular}

\end{table}

\begin{table}[H]
\centering
\caption{The accuracy of KNN on letter dataset.}
\label{tab:tab_knn_letter}
\begin{tabular}{lr}
\toprule
{} &  Accuracy KNN on letter dataset \\
\midrule
The 1st run         &                           95.44 \\
The 2nd run         &                           95.61 \\
The 3rd run         &                           95.70 \\
The 4th run         &                           95.39 \\
The 5th run         &                           95.54 \\
The 6th run         &                           95.46 \\
The 7th run         &                           95.51 \\
The 8th run         &                           95.59 \\
The 9th run         &                           95.42 \\
The 10th run        &                           95.56 \\
Average accuracy:   &                           95.52 \\
Standard deviation: &                            0.09 \\
\bottomrule
\end{tabular}

\end{table}

\begin{table}[H]
\centering
\caption{The accuracy of KNN on mushroom dataset.}
\label{tab:tab_knn_mushroom}
\begin{tabular}{lr}
\toprule
{} &  Accuracy KNN on mushroom dataset \\
\midrule
The 1st run         &                             99.88 \\
The 2nd run         &                             99.88 \\
The 3rd run         &                             99.89 \\
The 4th run         &                             99.88 \\
The 5th run         &                             99.86 \\
The 6th run         &                             99.91 \\
The 7th run         &                             99.86 \\
The 8th run         &                             99.89 \\
The 9th run         &                             99.86 \\
The 10th run        &                             99.93 \\
Average accuracy:   &                             99.88 \\
Standard deviation: &                              0.02 \\
\bottomrule
\end{tabular}

\end{table}











\subsection{Naive Bayes algorithm}

 

It is a fast algorithm. In this algorithm, we must assume that the features are independent with a Gaussian distribution. As you can see, the algorithm had a great result on cancer and mushroom dataset, while for other datasets, the results were not good. The reason could be neither features are independent nor their distribution is Gaussian. This algorithm does not have any hyperparameters. Furthermore, this algorithm takes the numerical values and since Car and Mushroom dataset have the categorical attributes, I have converted them to numerical version by the replace method in Pandas library.    



\begin{table}[H]
\centering
\caption{The accuracy of Naive Bayes on cancer dataset.}
\label{tab:tab_NB_canc}
\begin{tabular}{lr}
\toprule
{} &  Accuracy Naive Bayes on cancer dataset \\
\midrule
The 1st run         &                                   96.32 \\
The 2nd run         &                                   96.03 \\
The 3rd run         &                                   96.47 \\
The 4th run         &                                   95.88 \\
The 5th run         &                                   96.18 \\
The 6th run         &                                   96.47 \\
The 7th run         &                                   96.47 \\
The 8th run         &                                   96.32 \\
The 9th run         &                                   96.18 \\
The 10th run        &                                   96.32 \\
Average accuracy:   &                                   96.26 \\
Standard deviation: &                                    0.19 \\
\bottomrule
\end{tabular}

\end{table}

\begin{table}[H]
\centering
\caption{The accuracy of Naive Bayes on cars dataset.}
\label{tab:tab_NB_cars}
\begin{tabular}{lr}
\toprule
{} &  Accuracy Naive Bayes on cars dataset \\
\midrule
The 1st run         &                                 82.09 \\
The 2nd run         &                                 81.80 \\
The 3rd run         &                                 82.20 \\
The 4th run         &                                 82.38 \\
The 5th run         &                                 82.26 \\
The 6th run         &                                 82.14 \\
The 7th run         &                                 82.49 \\
The 8th run         &                                 81.39 \\
The 9th run         &                                 81.91 \\
The 10th run        &                                 82.61 \\
Average accuracy:   &                                 82.13 \\
Standard deviation: &                                  0.34 \\
\bottomrule
\end{tabular}

\end{table}

\begin{table}[H]
\centering
\caption{The accuracy of Naive Bayes on ecol dataset.}
\label{tab:tab_NB_ecol}
\begin{tabular}{lr}
\toprule
{} &  Accuracy Naive Bayes on ecol dataset \\
\midrule
The 1st run         &                                 57.61 \\
The 2nd run         &                                 61.49 \\
The 3rd run         &                                 51.64 \\
The 4th run         &                                 57.01 \\
The 5th run         &                                 58.51 \\
The 6th run         &                                 54.33 \\
The 7th run         &                                 67.46 \\
The 8th run         &                                 61.49 \\
The 9th run         &                                 57.91 \\
The 10th run        &                                 49.25 \\
Average accuracy:   &                                 57.67 \\
Standard deviation: &                                  4.95 \\
\bottomrule
\end{tabular}

\end{table}

\begin{table}[H]
\centering
\caption{The accuracy of Naive Bayes on letter dataset.}
\label{tab:tab_NB_letter}
\begin{tabular}{lr}
\toprule
{} &  Accuracy Naive Bayes on letter dataset \\
\midrule
The 1st run         &                                   64.99 \\
The 2nd run         &                                   65.01 \\
The 3rd run         &                                   64.87 \\
The 4th run         &                                   64.92 \\
The 5th run         &                                   64.83 \\
The 6th run         &                                   64.94 \\
The 7th run         &                                   64.95 \\
The 8th run         &                                   64.89 \\
The 9th run         &                                   64.90 \\
The 10th run        &                                   64.80 \\
Average accuracy:   &                                   64.91 \\
Standard deviation: &                                    0.06 \\
\bottomrule
\end{tabular}

\end{table}

\begin{table}[H]
\centering
\caption{The accuracy of Naive Bayes on mushroom dataset.}
\label{tab:tab_NB_mushroom}
\begin{tabular}{lr}
\toprule
{} &  Accuracy Naive Bayes on mushroom dataset \\
\midrule
The 1st run         &                                     91.86 \\
The 2nd run         &                                     91.91 \\
The 3rd run         &                                     91.65 \\
The 4th run         &                                     91.79 \\
The 5th run         &                                     91.72 \\
The 6th run         &                                     91.84 \\
The 7th run         &                                     91.86 \\
The 8th run         &                                     91.88 \\
The 9th run         &                                     91.44 \\
The 10th run        &                                     91.76 \\
Average accuracy:   &                                     91.77 \\
Standard deviation: &                                      0.14 \\
\bottomrule
\end{tabular}

\end{table}











\subsection{Iterative Dichotomiser 3 algorithm (ID3)}
The speed of building decision tree is relatively fast, the algorithm is simple, and the generated rules are easy to understand. Also it could not support continuous values. Therefore, all datasets were discretized before using this algorithm (see session \ref{ses:Discretization} for more detail). 


\begin{table}[H]
\centering
\caption{The accuracy of ID3 on mushroom dataset.}
\label{tab:tab_id3_mushroom}
\input{manuscript/src/tables/programing_project/Accuracy ID3 on mushroom dataset}
\end{table}

\begin{table}[H]
\centering
\caption{The accuracy of ID3 on cancer dataset.}
\label{tab:tab_id3_cancer}
\begin{tabular}{lr}
\toprule
{} &  Accuracy ID3 on cancer dataset \\
\midrule
The 1st run         &                           93.68 \\
The 2nd run         &                           93.09 \\
The 3rd run         &                           94.71 \\
The 4th run         &                           94.41 \\
The 5th run         &                           95.15 \\
The 6th run         &                           93.53 \\
The 7th run         &                           94.85 \\
The 8th run         &                           93.97 \\
The 9th run         &                           93.68 \\
The 10th run        &                           94.26 \\
Average accuracy:   &                           94.13 \\
Standard deviation: &                            0.62 \\
\bottomrule
\end{tabular}

\end{table}

\begin{table}[H]
\centering
\caption{The accuracy of ID3 on cars dataset.}
\label{tab:tab_id3_cars}
\begin{tabular}{lr}
\toprule
{} &  Accuracy ID3 on cars dataset \\
\midrule
The 1st run         &                         90.61 \\
The 2nd run         &                         90.09 \\
The 3rd run         &                         90.32 \\
The 4th run         &                         91.13 \\
The 5th run         &                         90.38 \\
The 6th run         &                         91.25 \\
The 7th run         &                         90.32 \\
The 8th run         &                         90.38 \\
The 9th run         &                         89.80 \\
The 10th run        &                         90.49 \\
Average accuracy:   &                         90.48 \\
Standard deviation: &                          0.41 \\
\bottomrule
\end{tabular}

\end{table}

\begin{table}[H]
\centering
\caption{The accuracy of ID3 on ecol dataset.}
\label{tab:tab_id3_ecol}
\begin{tabular}{lr}
\toprule
{} &  Accuracy ID3 on ecol dataset \\
\midrule
The 1st run         &                         62.39 \\
The 2nd run         &                         66.57 \\
The 3rd run         &                         65.97 \\
The 4th run         &                         65.37 \\
The 5th run         &                         62.99 \\
The 6th run         &                         63.28 \\
The 7th run         &                         65.37 \\
The 8th run         &                         64.78 \\
The 9th run         &                         66.27 \\
The 10th run        &                         65.97 \\
Average accuracy:   &                         64.90 \\
Standard deviation: &                          1.41 \\
\bottomrule
\end{tabular}

\end{table}

\begin{table}[H]
\centering
\caption{The accuracy of ID3 on letter dataset.}
\label{tab:tab_id3_letter}
\begin{tabular}{lr}
\toprule
{} &  Accuracy ID3 on letter dataset \\
\midrule
The 1st run         &                           75.51 \\
The 2nd run         &                           75.26 \\
The 3rd run         &                           75.56 \\
The 4th run         &                           75.64 \\
The 5th run         &                           75.90 \\
The 6th run         &                           75.54 \\
The 7th run         &                           75.43 \\
The 8th run         &                           75.38 \\
The 9th run         &                           75.54 \\
The 10th run        &                           75.50 \\
Average accuracy:   &                           75.53 \\
Standard deviation: &                            0.16 \\
\bottomrule
\end{tabular}

\end{table}
























\subsection{Random Forest algorithm}
Random Forest is less impacted by noise, because we train several decision trees. For this project, the number of trees was set to 5. Due to the fact that this algorithm is using ID3, we need to discretize data. 

\begin{table}[H]
\centering
\caption{The accuracy of Random Forest on cancer dataset.}
\label{tab:tab_rf_cancer}
\begin{tabular}{lr}
\toprule
{} &  Accuracy Random Forest on cancer dataset \\
\midrule
The 1st run         &                                     96.47 \\
The 2nd run         &                                     95.44 \\
The 3rd run         &                                     94.85 \\
The 4th run         &                                     95.29 \\
The 5th run         &                                     95.15 \\
The 6th run         &                                     94.41 \\
The 7th run         &                                     94.85 \\
The 8th run         &                                     95.74 \\
The 9th run         &                                     95.44 \\
The 10th run        &                                     95.00 \\
Average accuracy:   &                                     95.26 \\
Standard deviation: &                                      0.54 \\
\bottomrule
\end{tabular}

\end{table}


\begin{table}[H]
\centering
\caption{The accuracy of Random Forest on cars dataset.}
\label{tab:tab_rf_cars}
\begin{tabular}{lr}
\toprule
{} &  Accuracy Random Forest on cars dataset \\
\midrule
The 1st run         &                                   90.20 \\
The 2nd run         &                                   90.09 \\
The 3rd run         &                                   89.04 \\
The 4th run         &                                   90.61 \\
The 5th run         &                                   91.19 \\
The 6th run         &                                   90.49 \\
The 7th run         &                                   90.55 \\
The 8th run         &                                   89.74 \\
The 9th run         &                                   90.03 \\
The 10th run        &                                   90.72 \\
Average accuracy:   &                                   90.27 \\
Standard deviation: &                                    0.56 \\
\bottomrule
\end{tabular}

\end{table}

\begin{table}[H]
\centering
\caption{The accuracy of Random Forest on ecol dataset.}
\label{tab:tab_rf_ecol}
\begin{tabular}{lr}
\toprule
{} &  Accuracy Random Forest on ecol dataset \\
\midrule
The 1st run         &                                   67.16 \\
The 2nd run         &                                   67.46 \\
The 3rd run         &                                   70.45 \\
The 4th run         &                                   68.06 \\
The 5th run         &                                   64.78 \\
The 6th run         &                                   67.76 \\
The 7th run         &                                   68.66 \\
The 8th run         &                                   65.37 \\
The 9th run         &                                   64.18 \\
The 10th run        &                                   66.87 \\
Average accuracy:   &                                   67.07 \\
Standard deviation: &                                    1.79 \\
\bottomrule
\end{tabular}

\end{table}

\begin{table}[H]
\centering
\caption{The accuracy of Random Forest on letter dataset.}
\label{tab:tab_rf_letter}
\begin{tabular}{lr}
\toprule
{} &  Accuracy Random Forest on letter dataset \\
\midrule
The 1st run         &                                     76.29 \\
The 2nd run         &                                     76.21 \\
The 3rd run         &                                     76.21 \\
The 4th run         &                                     76.48 \\
The 5th run         &                                     76.64 \\
The 6th run         &                                     76.22 \\
The 7th run         &                                     76.32 \\
The 8th run         &                                     76.36 \\
The 9th run         &                                     76.19 \\
The 10th run        &                                     76.36 \\
Average accuracy:   &                                     76.33 \\
Standard deviation: &                                      0.14 \\
\bottomrule
\end{tabular}

\end{table}


\begin{table}[H]
\centering
\caption{The accuracy of Random Forest on mushroom dataset.}
\label{tab:tab_rf_mushroom}
\begin{tabular}{lr}
\toprule
{} &  Accuracy Random Forest on mushroom dataset \\
\midrule
The 1st run         &                                       100.0 \\
The 2nd run         &                                       100.0 \\
The 3rd run         &                                       100.0 \\
The 4th run         &                                       100.0 \\
The 5th run         &                                       100.0 \\
The 6th run         &                                       100.0 \\
The 7th run         &                                       100.0 \\
The 8th run         &                                       100.0 \\
The 9th run         &                                       100.0 \\
The 10th run        &                                       100.0 \\
Average accuracy:   &                                       100.0 \\
Standard deviation: &                                         0.0 \\
\bottomrule
\end{tabular}

\end{table}







\subsection{Adaboost algorithm}
Adaboost is easy to implement. Also, it could not overtrain since each node is divided into two branches. It can be implemented with other classifications. I have implemented it with a binary stump classifier. As a result, I could classify only cancer and mushroom datasets because these are only two classes. For this project, the number of weak classifiers considered 5. Also this algorithm could only support symbolizing values. Therefore, all datasets were discretized before using this algorithm (see session \ref{ses:Discretization} for more detail). 





\begin{table}[H]
\centering
\caption{The accuracy of Adaboost on cancer dataset.}
\label{tab:tab_Adaboost_cancer}
\begin{tabular}{lr}
\toprule
{} &  Accuracy Adaboost on cancer dataset \\
\midrule
The 1st run         &                                95.15 \\
The 2nd run         &                                95.00 \\
The 3rd run         &                                95.15 \\
The 4th run         &                                95.88 \\
The 5th run         &                                95.74 \\
The 6th run         &                                94.12 \\
The 7th run         &                                95.44 \\
The 8th run         &                                94.56 \\
The 9th run         &                                95.15 \\
The 10th run        &                                94.26 \\
Average accuracy:   &                                95.04 \\
Standard deviation: &                                 0.55 \\
\bottomrule
\end{tabular}

\end{table}

\begin{table}[H]
\centering
\caption{The accuracy of Adaboost on mushroom dataset.}
\label{tab:tab_Adaboost_mushroom}
\begin{tabular}{lr}
\toprule
{} &  Accuracy Adaboost on mushroom dataset \\
\midrule
The 1st run         &                                  98.87 \\
The 2nd run         &                                  98.87 \\
The 3rd run         &                                  98.87 \\
The 4th run         &                                  98.87 \\
The 5th run         &                                  98.87 \\
The 6th run         &                                  98.87 \\
The 7th run         &                                  98.87 \\
The 8th run         &                                  98.87 \\
The 9th run         &                                  98.87 \\
The 10th run        &                                  98.87 \\
Average accuracy:   &                                  98.87 \\
Standard deviation: &                                   0.00 \\
\bottomrule
\end{tabular}

\end{table}
%%%%%%%%%%%%%%%%%%%%%%%%%%%%%%%%%%%%%%%%%%%%%%%%%%%%%%%%%%%%%%%%%
%%%%%%%%%%%%%%%%%%%%%%%% Question 4 %%%%%%%%%%%%%%%%%%%%%%%%%%%%%
%%%%%%%%%%%%%%%%%%%%%%%%%%%%%%%%%%%%%%%%%%%%%%%%%%%%%%%%%%%%%%%%%
%\newpage
%\input{Questions/Ass1-Q4}













%\newpage
%\section*{REFERENCES}
%\label{sec:sec6}
%\printbibliography[heading=none]

%\bibliography{references}


%\newpage
%\section*{Appendix (codes)}
%\subsection*{The script of sun.py}



\end{document}